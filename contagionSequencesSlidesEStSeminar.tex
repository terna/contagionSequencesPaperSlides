\documentclass[8pt]{beamer}


\mode<presentation>
{

  \usetheme{Szeged} %use default if problems
  % or ... https://deic-web.uab.cat/~iblanes/beamer_gallery/index_by_theme.html
}

\usepackage{booktabs}
\usepackage{color}

\usepackage[english]{babel}
% or whatever

\usepackage[latin1]{inputenc}
% or whatever

\usepackage{times}
\usepackage[T1]{fontenc}
% Or whatever. Note that the encoding and the font should match. If T1
% does not look nice, try deleting the line with the fontenc.
\usepackage{nth}
\usepackage{xcolor}


\setbeamertemplate{caption}[numbered]


\title[S.I.s.a.R. Epidemic Model] % (optional, use only with long paper titles)
{How can ABM models become part of the policy-making process in times of emergencies - The S.I.s.a.R. Epidemic Model}

\author[] % (optional, use only with lots of authors)
{G.~Pescarmona\inst{1} \and P.~Terna\inst{2} \and A.~Acquadro\inst{1} \and P.~Pescarmona\inst{3} \and G.~Russo\inst{4}  
\and S.~Terna\inst{5}  }
% - Give the names in the same order as the appear in the paper.
% - Use the \inst{?} command only if the authors have different
%   affiliation.


\institute[] % (optional, but mostly needed)
{
  \inst{1}%
 University of Torino, Italy
  \and
  \inst{2}%
  University of Torino, Italy, retired \& Fondazione Collegio Carlo Alberto, Italy
 \and
  \inst{3}%
  University of Groningen, The Netherlands  
  \and
  \inst{4}%
  Centro Einaudi, Torino, Italy
  \and
  \inst{5}%
 tomorrowdata.io
  }
% - Use the \inst command only if there are several affiliations.
% - Keep it simple, no one is interested in your street address.


\date[] % (optional, should be abbreviation of conference name)
{Department of Economics and Statistics ``Cognetti de Martiis'' -  ESt seminar\\November 12, 2020}

\begin{document}

%%%%%%%%%%%%%%%%%%%%%%%%%%%%%%%%%%%%%%%%%%%%%%%%%%%%%%%%%
\begin{frame}
  \titlepage
\end{frame}

%%%%%%%%%%%%%%%%%%%%%%%%%%%%%%%%%%%%%%%%%%%%%%%%%%%%%%%%%
\begin{frame}{Outline}
  \tableofcontents
  % You might wish to add the option [pausesections]
\end{frame}

%%%%%%%%%%%%%%%%%%%%%%%%%%%%%%%%%%%%%%%%%%%%%%%%%%%%%%%%%
\section{Introduction}

%%%%%%%%%%%%%%%%%%%%%%%%%%%%%%%%%%%%%%%%%%%%%%%%%%%%%%%%%
\begin{frame}{Objectives of the model}

  \begin{itemize}
  \item
We propose an agent-based model to simulate the Covid-19 epidemic diffusion, with Susceptible, Infected, symptomatic, asymptomatic, and Recovered people: hence the name S.I.s.a.R. The scheme comes from S.I.R. models, with (i) infected agents categorized as symptomatic and asymptomatic and (ii) the places of contagion specified in a detailed way, thanks to agent-based modeling capabilities. 

 \item
The infection transmission is related to three factors: the infected person's characteristics and the susceptible one, plus those of the space in which contact occurs.

 \item
The model includes the structural data of Piedmont, an Italian region, but it can be readily calibrated for other areas. The model reproduces a realistic calendar (e.g., national or local government decisions), via its script interpreter.  

\bigskip

 \item
 
S.I.s.a.R. is at \url{https://terna.to.it/simul/SIsaR.html} with information on model construction, the draft of a paper also reporting results, and an online executable version of the simulation program, built using NetLogo.

 \item
 A short paper at \url{https://rofasss.org/2020/10/20/sisar/}.

 \end{itemize}
\end{frame}

%%%%%%%%%%%%%%%%%%%%%%%%%%%%%%%%%%%%%%%%%%%%%%%%%%%%%%%%%
\begin{frame}{The scale and the items}

\begin{itemize}

\item $1:1000$.

\item Contagions are related to; (i) the characteristics of infected persons, (ii) the fragility of susceptible ones, (iii) the peculiarity of the different places, (iv) the distances, (v) probabilities.

\item Houses.
\item Schools.
\item Hospitals.
\item Nursing homes,
\item Factories.

\end{itemize}

\end{frame}
%%%%%%%%%%%%%%%%%%%%%%%%%%%%%%%%%%%%%%%%%%%%%%%%%%%%%%%%%
\section{NetLogo model}

%%%%%%%%%%%%%%%%%%%%%%%%%%%%%%%%%%%%%%%%%%%%%%%%%%%%%%%%%
\subsection{A static view}

%%%%%%%%%%%%%%%%%%%%%%%%%%%%%%%%%%%%%%%%%%%%%%%%%%%%%%%%%
\begin{frame}{The interface and the information sheet}

\begin{figure}[H]
\center
\includegraphics[scale=0.28]{interface.png}

\caption{The interface} 
\label{interface}
\end{figure}

\end{frame}

%%%%%%%%%%%%%%%%%%%%%%%%%%%%%%%%%%%%%%%%%%%%%%%%%%%%%%%%%
\begin{frame}{The interface and the information sheet}

\begin{figure}[H]
\center
\includegraphics[scale=0.23]{info1.png}~~~\includegraphics[scale=0.23]{info2.png}

\caption{The information sheet, about  20 pages} 
\label{interface}
\end{figure}

\end{frame}

%%%%%%%%%%%%%%%%%%%%%%%%%%%%%%%%%%%%%%%%%%%%%%%%%%%%%%%%%
\subsection{A live view}

%%%%%%%%%%%%%%%%%%%%%%%%%%%%%%%%%%%%%%%%%%%%%%%%%%%%%%%%%
\begin{frame}{The interface}

\bigskip
A live look to the running model, also in a 3D view.

\end{frame}

%%%%%%%%%%%%%%%%%%%%%%%%%%%%%%%%%%%%%%%%%%%%%%%%%%%%%%%%%
\section{Contagions}

%%%%%%%%%%%%%%%%%%%%%%%%%%%%%%%%%%%%%%%%%%%%%%%%%%%%%%%%%
\subsection{The proposed technique}

%%%%%%%%%%%%%%%%%%%%%%%%%%%%%%%%%%%%%%%%%%%%%%%%%%%%%%%%%
\begin{frame}{Contagion representation}

  \begin{itemize}
  \item
The model allows analyzing the sequences of contagions in simulated epidemics, taking in account the places where they occur. 
  \item
We represent each infecting agent as a horizontal segment with a vertical connection to another agent receiving the infection. 
We represent the second agent via a further segment at an upper layer. 

  \item
With colors, line thickness, and styles, we display multiple data. 

  \item
This enables understanding at a glance how an epidemic episode is developing. In this way, it is easier to reason about countermeasures and, thus, to develop intervention policies.

  \end{itemize}
\end{frame}

%%%%%%%%%%%%%%%%%%%%%%%%%%%%%%%%%%%%%%%%%%%%%%%%%%%%%%%%%
\subsection{An introductory example}


%%%%%%%%%%%%%%%%%%%%%%%%%%%%%%%%%%%%%%%%%%%%%%%%%%%%%%%%%
\begin{frame}{An example}

\begin{figure}[H]
\center
\includegraphics[width=0.9\textwidth]{with8b40.png}% with control case 473323 474697 in SIsaR_0.9.4.1 experiments 2 seeds with control-table_10000.csv, file withControl_473323_474697.csv
\caption{A case with containment measures, first 40 infections: workplaces (brown) and nursing homes (orange) strictly interweaving}
\label{workplacesNursingHomes}
\end{figure}


\end{frame}

%%%%%%%%%%%%%%%%%%%%%%%%%%%%%%%%%%%%%%%%%%%%%%%%%%%%%%%%%
\begin{frame}{Same example, more cases}

\begin{figure}[H]
\center
\includegraphics[width=0.9\textwidth]{with8a.png}% with control case 473323 474697 in SIsaR_0.9.4.1 experiments 2 seeds with control-table_10000.csv, file withControl_473323_474697.csv
\caption{A Case with containment measures, the whole epidemics: workplaces (brown) and nursing homes (orange) and then houses (cyan), with a bridge connecting two waves}
\label{workplacesNursingHomes}
\end{figure}


\end{frame}

%%%%%%%%%%%%%%%%%%%%%%%%%%%%%%%%%%%%%%%%%%%%%%%%%%%%%%%%%
\begin{frame}{Other examples (i) on the left, an epidemic without containment measures; (ii) on the right, an epidemic with basic non-pharmaceutical containment measures}

\begin{figure}[H]
\center
\includegraphics[scale=0.12]{no4b.png}~~~~~~~~~~~\includegraphics[scale=0.12]{with7b.png} 

\center
\includegraphics[scale=0.12]{no4a.png}~~~~~~~~~~~\includegraphics[scale=0.12]{with7a.png} \\
\caption{Two cases with initial and full periods} 
\label{twocases}
\end{figure}

\end{frame}

%%%%%%%%%%%%%%%%%%%%%%%%%%%%%%%%%%%%%%%%%%%%%%%%%%%%%%%%%
\subsection{A significant sequence}

%%%%%%%%%%%%%%%%%%%%%%%%%%%%%%%%%%%%%%%%%%%%%%%%%%%%%%%%%
\begin{frame}{}

A contagion sequence suggesting policies: in Fig. \ref{fourSequences} we can look both at the places where contagions occur and at the dynamics emerging with different levels of intervention. 

\begin{figure}[H]
\center
\includegraphics[scale=0.105]{withShort1.png}~~~~~~~~~~~\includegraphics[scale=0.105]{withShort1A.png} 

\center
\includegraphics[scale=0.105]{withShort1A200.png}~~~~~~~~~~~\includegraphics[scale=0.105]{withShort1B.png} \\
\caption{(\emph{top left}) an epidemic with regular containment measures, showing a highly significant effect of workplaces (brown);
 (\emph{top right}) the effects of stopping fragile workers at day 20, with a positive result, but home contagions (cyan) keep alive the pandemic, exploding again in workplaces (brown); (\emph{bottom left}) the same analyzing the first 200 infections with evidence of the event around day 110 with the new phase due to a unique asymptomatic worker, and (\emph{bottom right}) stopping fragile workers and any case of fragility at day 15, also isolating nursing homes} 
\label{fourSequences}
\end{figure}

\end{frame}

%%%%%%%%%%%%%%%%%%%%%%%%%%%%%%%%%%%%%%%%%%%%%%%%%%%%%%%%%
\section{Batches}

%%%%%%%%%%%%%%%%%%%%%%%%%%%%%%%%%%%%%%%%%%%%%%%%%%%%%%%%%
\subsection{Batches}

%%%%%%%%%%%%%%%%%%%%%%%%%%%%%%%%%%%%%%%%%%%%%%%%%%%%%%%%%
\begin{frame}{Batches}

  \begin{itemize}
  \item

We explore systematically the introduction of factual, counterfactual, and prospective interventions to control the spread of the contagions. 

  \item
Each simulation run---whose length coincides with the disappearance of symptomatic or asymptomatic contagion cases---is a datum in a wide scenario of variability in time and effects.   
  
  \item
  
Consequently, we need to represent compactly the results  emerging from batches of repetitions, to compare the  consequences of the basic assumptions adopted for each specific batch.

 \item
We used blocs of one thousand repetitions. Besides summarizing the results with the usual statistical indicators, we adopted the technique of the heat-maps.

\end{itemize}
\end{frame}

%%%%%%%%%%%%%%%%%%%%%%%%%%%%%%%%%%%%%%%%%%%%%%%%%%%%%%%%%
\subsection{Two heat-maps}

%%%%%%%%%%%%%%%%%%%%%%%%%%%%%%%%%%%%%%%%%%%%%%%%%%%%%%%%%
\begin{frame}{Two quite different heat-maps for the Piedmont region}

In Fig. \ref{2HM} we have two heat-maps reporting the duration of each simulated epidemic in the $x$ axis and the number of the symptomatic, asymptomatic, and deceased agents in the $y$ axis. 1,000 runs in both cases.

\begin{figure}[H]
\center
\includegraphics[scale=0.3]{HM30_readRunResults1k_noControl_plusHMlog.png}~~~\includegraphics[scale=0.3]{HM30_readRunResults1k_basicControl_schoolOpenSept_plusHMlog.png} 

\caption{(\emph{on the left}) Epidemics without containment measures;
 (\emph{on the right}) Epidemics with basic non-pharmaceutical containment measures, schools open in September 2020} 
\label{2HM}
\end{figure}


\end{frame}


%%%%%%%%%%%%%%%%%%%%%%%%%%%%%%%%%%%%%%%%%%%%%%%%%%%%%%%%%
\section{Policies}


%%%%%%%%%%%%%%%%%%%%%%%%%%%%%%%%%%%%%%%%%%%%%%%%%%%%%%%%%
\begin{frame}{Different Intervention: policies and results}

\begin{table}[H]
\center
\footnotesize
%\small
\begin{tabular}{lrrrrr}
\toprule
Scenarios                              &  total sym.  & total sym.,       & days~~~~  \\
{}                                           &                    & asympt., deceased   \\                               
\midrule
1.~no control                       &  {\color{red}851.12}~     &  {\color{red}2253.48}         &  340.10~  \\
                                            &  (288.52)    &  (767.58)         &  (110.21) \\
\midrule
2.~basic controls, no           &   {\color{blue}158.55}~    &  {\color{blue}416.98}~         &  196.97~  \\
 school in Sep 2020            &   (174.10)     &  (462.94)        &  (131.18) \\
\midrule
3.~basic controls, \emph{schools}   &   {\color{blue}153.71}~    &       {\color{blue}409.73}~       &  199.35~  \\
 \emph{open} in Sep 2020                &  (168.55)   &     (454.12)        &  (129.00) \\
\midrule
4.~basic controls, \textbf{stop  fragile workers},   &   {\color{orange}120.17}~   &      {\color{orange}334.68}~         &  181.10~   \\
 no  schools in Sep 2020                          &   (149.10)  &     (413.90)         &   (125.46) \\
\midrule
%5.~basic controls, nursing homes isol.,    &  150.53~     &     408.08~         &  201.76~   \\
%no schools in Sep 2020                           &   (172.48)    &     (467.54)         &  (138.15) \\
%\midrule
%6.~basic controls, stop  fragile people,     &   154.15~    &     408.50~         &  195.81~  \\
%no   schools in Sep 2020                          &   (170.22)    &      (456.08)        &  (129.52) \\
%\midrule
5.~basic controls, \textbf{stop f. workers \&  f. people \&}    & \textbf{{\color{orange}105.63}}~    &  \textbf{{\color{orange}302.62}}~      &  174.39~   \\
\textbf{ n. h. isol}., no sch, Sep.                                              &   (134.80)   &     (382.14)          &  (121.82) \\
 \midrule
6.~b. controls, stop f. workers \&  f. people \& nur. h. isol.,  &  124.10~    &    397.05~           &  200.31~    \\
\& \textbf{factories op.}, no sch. Sep.                                               &  (132.42)   &    (399.64)           &  (121.46) \\
\midrule
7.~b. controls, stop f. workers  \&  f. people \& nur. h. isol.,   &  116.55~   &    374.68~           &  195.28~    \\
 \& \textbf{factories op., sch. open} Sep.                                            &  (130.91) &     (394.66)           &  (119.33) \\
\bottomrule
\end{tabular}
\caption{Report of the key results, with mean and (std)}
\label{keyResultsT}
\end{table}



\end{frame}

%%%%%%%%%%%%%%%%%%%%%%%%%%%%%%%%%%%%%%%%%%%%%%%%%%%%%%%%%
\section{Current situation}

%%%%%%%%%%%%%%%%%%%%%%%%%%%%%%%%%%%%%%%%%%%%%%%%%%%%%%%%%
\begin{frame}{Where we were}

\begin{figure}[H]
\center
\includegraphics[scale=0.55]{WhereWeWere.png}

\caption{$\rightarrow$ the cell of epidemics concluded in May (less than 120k symptomatic + asymptomatic)} 
\label{WhereWeWere}
\end{figure}
\end{frame}

%%%%%%%%%%%%%%%%%%%%%%%%%%%%%%%%%%%%%%%%%%%%%%%%%%%%%%%%%
\begin{frame}{Where we are}

\begin{figure}[H]
\center
\includegraphics[scale=0.55]{WhereWeAre.png}

\caption{$\rightarrow$ where (and when) the second wave will end?} 
\label{WhereWeAre}
\end{figure}

\end{frame}

%%%%%%%%%%%%%%%%%%%%%%%%%%%%%%%%%%%%%%%%%%%%%%%%%%%%%%%%%
\section{Current data}

%%%%%%%%%%%%%%%%%%%%%%%%%%%%%%%%%%%%%%%%%%%%%%%%%%%%%%%%%
\begin{frame}{Raw data}

Sources: \url{https://terna.to.it/datiProtezioneCivile.html}  based on \url{http://www.protezionecivile.gov.it} data, daily updated.

\begin{figure}[H]
\center
\includegraphics[scale=0.26]{rawCurrent.png}

\caption{Raw data introduces the cases of screening analysis, making tests} 
\label{rawCurrent}
\end{figure}

\end{frame}

%%%%%%%%%%%%%%%%%%%%%%%%%%%%%%%%%%%%%%%%%%%%%%%%%%%%%%%%%
\begin{frame}{Comparable current data}

\begin{figure}[H]
\center
\includegraphics[scale=0.26]{comparableCurrent.png}

\caption{Dashed lines (cumulative values and daily stocks) are corrected deducing the number of screening cases (as cumulative values or 14 days differences} 
\label{comparableCurrent}
\end{figure}

\end{frame}

%%%%%%%%%%%%%%%%%%%%%%%%%%%%%%%%%%%%%%%%%%%%%%%%%%%%%%%%%
\section{A new model}

%%%%%%%%%%%%%%%%%%%%%%%%%%%%%%%%%%%%%%%%%%%%%%%%%%%%%%%%%
\begin{frame}{A new model: the map}

\begin{figure}[H]
\center
\includegraphics[scale=0.25]{Piem1.png}~~~~~~~~~\includegraphics[scale=0.25]{Piem2.png} 

\caption{3D Piedmont} 
\label{Piem}
\end{figure}

\end{frame}

%%%%%%%%%%%%%%%%%%%%%%%%%%%%%%%%%%%%%%%%%%%%%%%%%%%%%%%%%
\begin{frame}{A new model: the scale and the items}

\begin{itemize}

\item $1:100$.

\item Contagion engine.

\item Houses.
\item Schools.
\item Hospitals.
\item Nursing homes,
\item Factories.
\color{red}
\item Transportations.
\item Aggregation places: happy hours, night life, sport stadiums, discotheques, \ldots

\end{itemize}

\end{frame}

%%%%%%%%%%%%%%%%%%%%%%%%%%%%%%%%%%%%%%%%%%%%%%%%%%%%%%%%%
\begin{frame}{The tool: S.L.A.P.P.}

Scientific advertising: \url{https://terna.github.io/SLAPP/}

\begin{figure}[H]
\center
\includegraphics[scale=0.26]{SLAPP.png}

\caption{Swarm-Like Agent Protocol in Python} 
\label{SLAPP}
\end{figure}

\end{frame}
%%%%%%%%%%%%%%%%%%%%%%%%%%%%%%%%%%%%%%%%%%%%%%%%%%%%%%%%%
\section{Final remarks}

%%%%%%%%%%%%%%%%%%%%%%%%%%%%%%%%%%%%%%%%%%%%%%%%%%%%%%%%%
\begin{frame}{A few considerations}

\begin{itemize}

\item The model is a tool for comparative analyses, not forecasting (the enormous standard deviation values are intrinsic to the problem).

\bigskip                
\item \textit{How can your work be adapted to (or is relevant/useful for) another disease, crisis, context, \ldots}

\bigskip
\item The model is highly parametric and more it will be.

\bigskip
\item New crisis calling for immediate simulation could take a substantial advantage from the parametric structure of the model.

\end{itemize}

 \bigskip
 The slides are at \url{https://terna.to.it/PietroTernaEStSeminar.pdf}.
 
\end{frame}

\end{document}