\documentclass[8pt]{beamer}


\mode<presentation>
{
\usetheme{Singapore} %use default if problems or Szeged
  % or ... https://deic-web.uab.cat/~iblanes/beamer_gallery/index_by_theme.html
}
\setbeamertemplate{footline}[frame number]
\usepackage{booktabs}
\usepackage{color}

\usepackage[english]{babel}
% or whatever

\usepackage[latin1]{inputenc}
% or whatever

\usepackage{times}
\usepackage[T1]{fontenc}
% Or whatever. Note that the encoding and the font should match. If T1
% does not look nice, try deleting the line with the fontenc.
\usepackage[super]{nth}
\usepackage{xcolor}
\usepackage{relsize} %large math 

\usepackage{graphicx} % to insert the logo

\usepackage[font=scriptsize,skip=1pt]{caption}

\setbeamertemplate{caption}[numbered]

\title[S.I.s.a.R. Model] % (optional, use only with long paper titles)
{An Agent-Based Model to simulate the Covid-19 epidemic diffusion, with the use of Genetic Algorithms to optimize vaccinations}

\author[] % (optional, use only with lots of authors)
{G.~Pescarmona\inst{1} \and P.~Terna\inst{2} \and A.~Acquadro\inst{1} \and P.~Pescarmona\inst{3} \and G.~Russo\inst{4}  
\and S.~Terna\inst{5}  }
% - Give the names in the same order as the appear in the paper.
% - Use the \inst{?} command only if the authors have different
%   affiliation.


\institute[] % (optional, but mostly needed)
{
  \inst{1}%
 University of Torino, Italy
  \and
  \inst{2}%
  University of Torino, Italy, retired \& Fondazione Collegio Carlo Alberto, Honorary Fellow, Italy
 \and
  \inst{3}%
  University of Groningen, The Netherlands  
  \and
  \inst{4}%
  Centro Einaudi, Torino, Italy
  \and
  \inst{5}%
 tomorrowdata.io
  }
% - Use the \inst command only if there are several affiliations.
% - Keep it simple, no one is interested in your street address.


\date[] % (optional, should be abbreviation of conference name)
{CCA -- March \nth{29}, 2021}

\begin{document}

%%%%%%%%%%%%%%%%%%%%%%%%%%%%%%%%%%%%%%%%%%%%%%%%%%%%%%%%%
\begin{frame}

\includegraphics[width=0.18\textwidth]{logo_unito.png}~~~\includegraphics[width=0.15\textwidth]{CCA_Logo.png}

  \titlepage
\end{frame}

%%%%%%%%%%%%%%%%%%%%%%%%%%%%%%%%%%%%%%%%%%%%%%%%%%%%%%%%%
\begin{frame}{Outline}
  \tableofcontents
  % You might wish to add the option [pausesections]
\end{frame}

%%%%%%%%%%%%%%%%%%%%%%%%%%%%%%%%%%%%%%%%%%%%%%%%%%%%%%%%%
\section{Introduction}

%%%%%%%%%%%%%%%%%%%%%%%%%%%%%%%%%%%%%%%%%%%%%%%%%%%%%%%%%
\begin{frame}{Objectives of the model}

  \begin{itemize}
  \item
We propose an agent-based model to simulate the Covid-19 epidemic diffusion, with Susceptible, Infected, symptomatic, asymptomatic, and Recovered people: hence the name S.I.s.a.R. The scheme comes from S.I.R. models, with (i) infected agents categorized as symptomatic and asymptomatic and (ii) the places of contagion specified in a detailed way, thanks to agent-based modeling capabilities. 

 \item
The infection transmission is related to three factors: the infected person's characteristics and those of the susceptible one, plus those of the space in which a contact occurs.

 \item
The model includes the structural data of Piedmont, but it can be readily calibrated for other areas. The model manages a realistic calendar (e.g., national or local government decisions), via a script interpreter.  


 \end{itemize}
\end{frame}

%%%%%%%%%%%%%%%%%%%%%%%%%%%%%%%%%%%%%%%%%%%%%%%%%%%%%%%%%
\begin{frame}{Tool and links}

  \begin{itemize}

\item We use NetLogo, at \url{https://ccl.northwestern.edu/netlogo/}.

 \item
 
S.I.s.a.R. is at \url{https://terna.to.it/simul/SIsaR.html} with information on model construction, the draft of a paper also reporting results, and an online executable version of the simulation program, built using NetLogo.

 \item
 A short paper is published at \url{https://rofasss.org/2020/10/20/sisar/}
 
 G. Pescarmona, P. Terna, A. Acquadro, P. Pescarmona, G. Russo, and S. Terna. How Can ABM Models Become Part of the Policy-Making Process in Times of Emergencies---The SISAR Epidemic Model. \emph{RofASSS}, 2020.

 \end{itemize}
\end{frame}

%%%%%%%%%%%%%%%%%%%%%%%%%%%%%%%%%%%%%%%%%%%%%%%%%%%%%%%%%
\begin{frame}{The scale and the items}

\begin{itemize}

\item $1:1000$.

\bigskip

\item Houses.
\item Schools.
\item Hospitals.
\item Nursing homes,
\item Factories.

\end{itemize}

\end{frame}
%%%%%%%%%%%%%%%%%%%%%%%%%%%%%%%%%%%%%%%%%%%%%%%%%%%%%%%%%
\section{The model}

%%%%%%%%%%%%%%%%%%%%%%%%%%%%%%%%%%%%%%%%%%%%%%%%%%%%%%%%%
\subsection{Overview}

%%%%%%%%%%%%%%%%%%%%%%%%%%%%%%%%%%%%%%%%%%%%%%%%%%%%%%%%%
\begin{frame}{The interface and the information sheet}

\begin{figure}[H]
\center
\includegraphics[scale=0.14]{interface2021.png}

\caption{The interface} 
\label{interface}
\end{figure}

\end{frame}

%%%%%%%%%%%%%%%%%%%%%%%%%%%%%%%%%%%%%%%%%%%%%%%%%%%%%%%%%
\begin{frame}{The interface and the information sheet}

\begin{figure}[H]
\center
\includegraphics[scale=0.23]{info1b.png}~~~\includegraphics[scale=0.23]{info2b.png}

\caption{The information sheet, about  20 pages} 
\label{interface}
\end{figure}

\end{frame}

%%%%%%%%%%%%%%%%%%%%%%%%%%%%%%%%%%%%%%%%%%%%%%%%%%%%%%%%%
\subsection{Details}

%%%%%%%%%%%%%%%%%%%%%%%%%%%%%%%%%%%%%%%%%%%%%%%%%%%%%%%%%
\begin{frame}{The world}

\begin{figure}[H]
\center
\includegraphics[scale=0.35]{world.png}

\caption{The world} 
\label{world}
\end{figure}

\end{frame}

%%%%%%%%%%%%%%%%%%%%%%%%%%%%%%%%%%%%%%%%%%%%%%%%%%%%%%%%%
\begin{frame}{The world 3D}

\begin{figure}[H]
\center
\includegraphics[scale=0.55]{world3D.png}

\caption{The world 3D} 
\label{world3D}
\end{figure}

\end{frame}

%%%%%%%%%%%%%%%%%%%%%%%%%%%%%%%%%%%%%%%%%%%%%%%%%%%%%%%%%
\begin{frame}{The agents}

\begin{figure}[H]
\center
\includegraphics[scale=0.23]{person1.png}~~~~~~~~~~~~~~~~~~~~\includegraphics[scale=0.23]{person2.png}

\caption{Probes to different agents} 
\label{diffAgents}
\end{figure}

\end{frame}

%%%%%%%%%%%%%%%%%%%%%%%%%%%%%%%%%%%%%%%%%%%%%%%%%%%%%%%%%
\section{Contagions}

%%%%%%%%%%%%%%%%%%%%%%%%%%%%%%%%%%%%%%%%%%%%%%%%%%%%%%%%%
\subsection{Contagion representation}

%%%%%%%%%%%%%%%%%%%%%%%%%%%%%%%%%%%%%%%%%%%%%%%%%%%%%%%%%
\begin{frame}{Contagion representation}

  \begin{itemize}
  \item
The model allows analyzing the sequences of contagions in simulated epidemics, reporting the places where the contagion occur. 
  \item
We represent each infecting agent as a horizontal segment with a vertical connections to another agent receiving the infection. 
We represent the infected agents via further segments at an upper layer. 

  \item
With colors, line thickness, and styles, we display multiple information. 

  \item
This enables understanding at a glance how an epidemic episode is developing. In this way, it is easier to reason about countermeasures and, thus, to develop intervention policies.

  \end{itemize}
\end{frame}

%%%%%%%%%%%%%%%%%%%%%%%%%%%%%%%%%%%%%%%%%%%%%%%%%%%%%%%%%
\subsection{An introductory example}


%%%%%%%%%%%%%%%%%%%%%%%%%%%%%%%%%%%%%%%%%%%%%%%%%%%%%%%%%
\begin{frame}{An example (1/2)}

\begin{figure}[H]
\center
\includegraphics[width=0.9\textwidth]{with8b40.png}% with control case 473323 474697 in SIsaR_0.9.4.1 experiments 2 seeds with control-table_10000.csv, file withControl_473323_474697.csv
\caption{A case with containment measures, first 40 infections: workplaces (brown) and nursing homes (orange) strictly interweaving}
\label{workplacesNursingHomes}
\end{figure}
\end{frame}

%%%%%%%%%%%%%%%%%%%%%%%%%%%%%%%%%%%%%%%%%%%%%%%%%%%%%%%%%
\begin{frame}{An example (2/2), more contagions}

\begin{figure}[H]
\center
\includegraphics[width=0.9\textwidth]{with8a.png}% with control case 473323 474697 in SIsaR_0.9.4.1 experiments 2 seeds with control-table_10000.csv, file withControl_473323_474697.csv
\caption{A Case with containment measures, the whole epidemics: workplaces (brown) and nursing homes (orange) and then houses (cyan), with a bridge connecting two waves}
\label{workplacesNursingHomes}
\end{figure}


\end{frame}


%%%%%%%%%%%%%%%%%%%%%%%%%%%%%%%%%%%%%%%%%%%%%%%%%%%%%%%%%
\subsection{A significant sequence}

%%%%%%%%%%%%%%%%%%%%%%%%%%%%%%%%%%%%%%%%%%%%%%%%%%%%%%%%%
\begin{frame}{}

A contagion sequence suggesting policies: in Fig. \ref{fourSequences} we can look both at the places where contagions occur and at the dynamics emerging with different levels of intervention. 

\begin{figure}[H]
\center
\includegraphics[scale=0.105]{withShort1.png}~~~~~~~~~~~\includegraphics[scale=0.105]{withShort1A.png} 

\center
\includegraphics[scale=0.105]{withShort1A200.png}~~~~~~~~~~~\includegraphics[scale=0.105]{withShort1B.png} \\
\caption{(\emph{top left}) an epidemic with regular containment measures, showing a highly significant effect of workplaces (brown);
 (\emph{top right}) the effects of stopping fragile workers at day 20, with a positive result, but home contagions (cyan) keep alive the pandemic, exploding again in workplaces (brown); (\emph{bottom left}) the same analyzing the first 200 infections with evidence of the event around day 110 with the new phase due to a unique asymptomatic worker, and (\emph{bottom right}) stopping fragile workers and any case of fragility at day 15, also isolating nursing homes} 
\label{fourSequences}
\end{figure}

\end{frame}

%%%%%%%%%%%%%%%%%%%%%%%%%%%%%%%%%%%%%%%%%%%%%%%%%%%%%%%%%
\section{Exploring cases}

%%%%%%%%%%%%%%%%%%%%%%%%%%%%%%%%%%%%%%%%%%%%%%%%%%%%%%%%%
\subsection{Simulation batches}

%%%%%%%%%%%%%%%%%%%%%%%%%%%%%%%%%%%%%%%%%%%%%%%%%%%%%%%%%
\begin{frame}{Simulation batches}

  \begin{itemize}
  \item

We explore systematically the introduction of factual, counterfactual, and prospective interventions to control the spread of the contagions. 

  \item
Each simulation run---whose length coincides with the disappearance of symptomatic or asymptomatic contagion cases---is a datum in a wide scenario of variability in time and effects.   
  
  \item
Consequently, we need to represent compactly the results  emerging from batches of simulation repetitions, to compare the  consequences of the basic assumptions adopted for each specific batch.

 \item
We use blocs of ten thousand repetitions. Besides summarizing the results with the usual statistical indicators, we adopt the technique of the heatmaps.

\item
Each heatmap reports the duration of each simulated epidemic in the $x$ axis and the number of the symptomatic, asymptomatic, and deceased agents in the $y$ axis. The $z$ axis is represented by the colors, in logarithmic scale. 

\item
In our batches we have 10,000 runs.

\end{itemize}
\end{frame}

%%%%%%%%%%%%%%%%%%%%%%%%%%%%%%%%%%%%%%%%%%%%%%%%%%%%%%%%%
\subsection{Epidemics without and with control}

%%%%%%%%%%%%%%%%%%%%%%%%%%%%%%%%%%%%%%%%%%%%%%%%%%%%%%%%%
\begin{frame}{10,000 epidemics without control in Piedmont}

% readRunResults10kStableSeedsCPoints_noControl_ChangingWorld_plusHMlog

\begin{table}[H]
\center
\tiny

\begin{tabular}{lrrr}
\toprule
{} &  symptomatic &  totalInfected\&Deceased &  duration \\
\midrule
count &     10000.00 &                10000.00 &  10000.00 \\
mean  &       969.46 &                 2500.45 &    303.10 \\
std   &       308.80 &                  802.88 &     93.50 \\
\bottomrule
\end{tabular}

\label{noCTab}
%\caption{a caption}
\end{table}

\begin{figure}[H]
\center
\includegraphics[scale=0.22]{10kNoControl.png}
\caption{Without non-pharmaceutical containment measures} 
\label{noC}
\end{figure}

\end{frame}


%%%%%%%%%%%%%%%%%%%%%%%%%%%%%%%%%%%%%%%%%%%%%%%%%%%%%%%%%
\begin{frame}{10,000 epidemic with basic control in Piedmont, first wave}

% readRunResults10kStableSeedsCPoints_basicControlB_schoolOpenSeptChangingWorld_plusHMlog

\begin{table}[H]
\center
\tiny

\begin{tabular}{lrrr}
\toprule
{} &  symptomatic &  totalInfected\&Deceased &  duration \\
\midrule
count &     10000.00 &                10000.00 &  10000.00 \\
mean  &       344.22 &                  851.64 &    277.93 \\
std   &       368.49 &                  916.41 &    213.48 \\
\bottomrule
\end{tabular}

\label{basicCTab}
%\caption{a caption}
\end{table}

\begin{figure}[H]
\center
\includegraphics[scale=0.22]{10kBasicC.png}
\caption{First wave with non-pharmaceutical containment measures} 
\label{basicC}
\end{figure}

\end{frame}


%%%%%%%%%%%%%%%%%%%%%%%%%%%%%%%%%%%%%%%%%%%%%%%%%%%%%%%%%
\subsection{Actual data}

%%%%%%%%%%%%%%%%%%%%%%%%%%%%%%%%%%%%%%%%%%%%%%%%%%%%%%%%%
\begin{frame}{Key points}

\begin{figure}[H]
\center
\includegraphics[scale=0.25]{andamento900annotato.jpg}
\caption{keyPoints} 
\label{Key points}
\end{figure}


\end{frame}

%%%%%%%%%%%%%%%%%%%%%%%%%%%%%%%%%%%%%%%%%%%%%%%%%%%%%%%%%
\begin{frame}{Non omogeneous data}

\begin{itemize}

\item From Civil Protection Department web site \url{http://www.protezionecivile.it/web/guest/department} we find the
 repository \url{https://github.com/pcm-dpc/COVID-19}.
 
 \item In the first wave we had uniquely data about symptomatic infected people, but from October 2021 data are mixed.
 
 \item From the above \emph{git}  repository in October and November we had ``Positive cases emerged from clinical activity'', unfortunately now reported as ``No longer populated'' (from the end of November, my observation) and ``Positive cases emerging from surveys and tests, planned at national or regional level'', again ``No longer populated'' (from the end of November, my observation).
 
 \item Using those two series, it was possible to estimate a subdivision between symptomatic and asymptomatic cases, which is no longer possible.
 
\end{itemize}

\end{frame}




%%%%%%%%%%%%%%%%%%%%%%%%%%%%%%%%%%%%%%%%%%%%%%%%%%%%%%%%%
\begin{frame}{Updated series (close to the end of March)}

\begin{figure}[H]
\center
\includegraphics[scale=0.35]{andamento900.jpg}
\caption{Data for Piedmont} 
\label{dataP}
\end{figure}


\end{frame}




%%%%%%%%%%%%%%%%%%%%%%%%%%%%%%%%%%%%%%%%%%%%%%%%%%%%%%%%%
\subsection{Second wave}

%1%%%%%%%%%%%%%%%%%%%%%%%%%%%%%%%%%%%%%%%%%%%%%%%%%%%%%%%%
\begin{frame}{Spontaneous second wave, without specific measures}

% selectResults10kStableSeedsCPoints_basicControlB_schoolOpenSeptChangingWorld_plusHMlog,ipynb
% using 10kCtrl1.csv from SIsaR_0.9.5.4.1trials10kCtrl1.nlogo

\textbf{170} {\tiny epidemics stable in Summer 2020 out of 10,000, rule: at Jun~1,~20 select if sym. (10, 70], actual v. 33.3 \& at Sep~20,~20 select if sym. (20, 90], actual value 37.5;} \textbf{140} {\tiny at Dec~15,~20, rule: sym.+asym.>Sep~20,~20, actual value: 200.0.}

\begin{figure}[H]
\center
\includegraphics[scale=0.17]{10kSpontWave2.png}
\caption{First wave with non-pharmaceutical containment measures, spontaneous second wave, without specific measures}
\label{selSpontWave2}
\end{figure}

%\vspace{-0.4cm}

\begin{table}[H]
\center
\tiny
\begin{tabular}{p{0.3cm}p{0.3cm}p{0.3cm}p{0.3cm}p{0.3cm}p{0.3cm}p{0.3cm}p{0.3cm}p{0.3cm}p{0.3cm}p{0.3cm}p{0.3cm}p{0.3cm}p{0.4cm}}
\toprule
(1000) &  Jun~1,~20 & &  Sep~9,~20 & & Dec~15,~20 & & Feb~1,~21 & & May~1,~21 & & Dec~15,~20~~~to~~~end  \\
{} &  sym. &  all &  sympt. &  totalInf. &  sympt. &  totalInf. &  sympt. &  totalInf. &  sympt. &  totalInf. &  sympt. &  totalInf.  & days\\
\midrule
count &    170.0 &                      170.0 &    170.0 &                      170.0 &    140.0 &                      140.0 &    131.0 &                      131.0 &    128.0 &                      128.0 &               140.0 &                   140.0 &  140.0 \\
mean  &     37.9 &                      100.2 &     60.4 &                      159.3 &    \textbf{248.4} &                      \textbf{648.7} &    \textbf{432.2} &                     \textbf{1109.5} &   \textbf{656.3} &                     \textbf{1655.5} &              701.1 &                  1757.9 &  594.2 \\
std   &     16.4 &                       61.0 &     19.6 &                       71.7 &    167.4 &                      424.3 &    220.4 &                      538.4 &    215.4 &                      513.3 &               246.4 &                   599.7 &  118.9 \\
\bottomrule
\end{tabular}

\label{selSpontWave2Tab}
%\caption{a caption}
\end{table}


\end{frame}

%2%%%%%%%%%%%%%%%%%%%%%%%%%%%%%%%%%%%%%%%%%%%%%%%%%%%%%%%%
\begin{frame}{Second w., new infections from outside, without specific measures}

% selectResults10kStableSeedsCPoints_basicControlB_schoolOpenSeptChangingWorldNewStart_plusHMlog.ipynb
% using 10kCtrl1NStart.csv from SIsaR_0.9.5.4.1trials10kCtrl1NStart.nlogo

\textbf{1407} {\tiny epidemics stable in Summer 2020 out of 10,000, rule: at Jun~1,~20 select if sym. (10, 70], actual v. 33.3 \& at Sep~20,~20 select if sym. (20, 90], actual value 37.5;} \textbf{1044} {\tiny at Dec~15,~20, rule: sym.+asym.>Sep~20,~20, actual value: 200.0.}

\begin{figure}[H]
\center
\includegraphics[scale=0.17]{10kForceWave2.png}
\caption{First wave with non-pharmaceutical containment measures, forcing the second wave, without specific measures}
\label{selForceWave2}
\end{figure}

%\vspace{-0.4cm}

\begin{table}[H]
\center
\tiny
\begin{tabular}{p{0.3cm}p{0.3cm}p{0.3cm}p{0.3cm}p{0.3cm}p{0.3cm}p{0.3cm}p{0.3cm}p{0.3cm}p{0.3cm}p{0.3cm}p{0.3cm}p{0.3cm}p{0.4cm}}
\toprule
(1000) &  Jun~1,~20 & &  Sep~9,~20 & & Dec~15,~20 & & Feb~1,~21 & & May~1,~21 & & Dec~15,~20~~~to~~~end   \\
{} &  sym. &  all &  sympt. &  totalInf. &  sympt. &  totalInf. &  sympt. &  totalInf. &  sympt. &  totalInf. &  sympt. &  totalInf.  & days\\
\midrule
count &   1407.0 &                     1407.0 &   1407.0 &                     1407.0 &   1044.0 &                     1044.0 &   1005.0 &                     1005.0 &    980.0 &                      980.0 &              1044.0 &                  1044.0 & 1044.0 \\
mean  &     35.6 &                       72.7 &     40.0 &                       84.1 &    \textbf{180.4} &                      \textbf{462.1} &    \textbf{354.1} &                      \textbf{900.4} &    \textbf{623.8} &                     \textbf{1563.3} &               726.6 &                  1810.9 &  620.9 \\
std   &     14.1 &                       42.6 &     16.7 &                       52.8 &    134.6 &                      354.6 &    213.8 &                      535.4 &    217.9 &                      527.0 &               221.9 &                   544.0 &  110.8 \\
\bottomrule
\end{tabular}

\label{selForceWave2Tab}
%\caption{a caption}
\end{table}


\end{frame}

%3%%%%%%%%%%%%%%%%%%%%%%%%%%%%%%%%%%%%%%%%%%%%%%%%%%%%%%%%
\begin{frame}{Second w., new infections from outside, with new specific measures}

% selectResults10kStableSeedsCPoints_basicControlB_schoolOpenSeptOctMarControlChangingWorldNewStart_plusHMlog.ipynb
% using 10kCtrl1NStartCtrl2M.csv from SIsaR_0.9.5.4.1trials10kCtrl1NStartCtrl2M.nlogo

\textbf{1407} {\tiny epidemics stable in Summer 2020 out of 10,000, rule: at Jun~1,~20 select if sym. (10, 70], actual v. 33.3 \& at Sep~20,~20 select if sym. (20, 90], actual value 37.5;} \textbf{874} {\tiny at Dec~15,~20, rule: sym.+asym.>Sep~20,~20, actual value: 200.0.}

\begin{figure}[H]
\center
\includegraphics[scale=0.17]{10kForceWave2Contr2.png}
\caption{First wave with non-ph.containment measures, forcing the second wave, \textbf{with new specific non-ph. containment measures}}
\label{selForceWave2Contr2}
\end{figure}

%\vspace{-0.4cm}

\begin{table}[H]
\center
\tiny
\begin{tabular}{p{0.3cm}p{0.3cm}p{0.3cm}p{0.3cm}p{0.3cm}p{0.3cm}p{0.3cm}p{0.3cm}p{0.3cm}p{0.3cm}p{0.3cm}p{0.3cm}p{0.3cm}p{0.4cm}}
\toprule
(1000) &  Jun~1,~20 & &  Sep~9,~20 & & Dec~15,~20 & & Feb~1,~21 & & May~1,~21 & & Dec~15,~20~~~to~~~end   \\
{} &  sym. &  all &  sympt. &  totalInf. &  sympt. &  totalInf. &  sympt. &  totalInf. &  sympt. &  totalInf. &  sympt. &  totalInf.  & days\\
\midrule
count &   1407.0 &                     1407.0 &   1407.0 &                     1407.0 &    874.0 &                      874.0 &    719.0 &                      719.0 &    523.0 &                      523.0 &              874.0 &                   874.0 &  874.0 \\
mean  &     35.6 &                       72.7 &     40.0 &                       84.1 &    \textbf{130.0} &                      \textbf{340}.6 &    \textbf{194.4} &                      \textbf{512.8} &    \textbf{295.7} &                      \textbf{791.2} &               252.7 &                   666.4 &  494.1 \\
std   &     14.1 &                       42.6 &     16.7 &                       52.8 &     83.9 &                      232.6 &    104.1 &                      276.9 &    119.1 &                      300.6 &               156.8 &                   416.4 &  122.7 \\
\bottomrule
\end{tabular}

\label{selSpontWave2Contr2Tab}
%\caption{a caption}
\end{table}


\end{frame}

%%%%%%%%%%%%%%%%%%%%%%%%%%%%%%%%%%%%%%%%%%%%%%%%%%%%%%%%%
\begin{frame}{Time factor}

\begin{figure}[H]
\center
\includegraphics[scale=0.19]{RtEstimation.jpg}
\caption{In blue the $R_t$ values as reported by the Istituto Superiore di Sanit\`{a} and in red the calculation published regularly at \url{https://mondoeconomico.eu} by Stefano Terna\footnote{Methodology: \url{https://github.com/tomorrowdata/COVID-19/blob/main/notebooks/Rt_on_italian_national_data.ipynb}}.}
\label{Rt}
\end{figure}


\end{frame}

%4%%%%%%%%%%%%%%%%%%%%%%%%%%%%%%%%%%%%%%%%%%%%%%%%%%%%%%%%
\begin{frame}{Second w., new infect. from outside, with new specific meas. -20 days\footnote{Anticipation limit Oct 5.}}

% selectResults10kStableSeedsCPoints_basicControlB_schoolOpenSeptOctMar-20ControlChangingWorldNewStart_plusHMlog.ipynb
% using 10kCtrl1NStartCtrl2M-20.csv from SIsaR_0.9.5.4.1trials10kCtrl1NStartCtrl2M-20.nlogo

\textbf{1407} {\tiny epidemics stable in Summer 2020 out of 10,000, rule: at Jun~1,~20 select if sym. (10, 70], actual v. 33.3 \& at Sep~20,~20 select if sym. (20, 90], actual value 37.5;} \textbf{769} {\tiny at Dec~15,~20, rule: sym.+asym.>Sep~20,~20, actual value: 200.0.}

\begin{figure}[H]
\center
\includegraphics[scale=0.17]{10kForceWave2Contr2M-20.png}
\caption{First wave with non-ph.cont. meas., forcing the second wave, \textbf{with new specific non-ph. cont. meas., 20 day anticipation}}
\label{selForceWave2Contr2M-20}
\end{figure}

%\vspace{-0.4cm}

\begin{table}[H]
\center
\tiny
\begin{tabular}{p{0.3cm}p{0.3cm}p{0.3cm}p{0.3cm}p{0.3cm}p{0.3cm}p{0.3cm}p{0.3cm}p{0.3cm}p{0.3cm}p{0.3cm}p{0.3cm}p{0.3cm}p{0.4cm}}
\toprule
(1000) &  Jun~1,~20 & &  Sep~9,~20 & & Dec~15,~20 & & Feb~1,~21 & & May~1,~21 & & Dec~15,~20~~~to~~~end   \\
{} &  sym. &  all &  sympt. &  totalInf. &  sympt. &  totalInf. &  sympt. &  totalInf. &  sympt. &  totalInf. &  sympt. &  totalInf.  & days\\
\midrule
count &   1407.0 &                     1407.0 &   1407.0 &                     1407.0 &    769.0 &                      769.0 &    637.0 &                      637.0 &    471.0 &                      471.0 &              769.0 &                   769.0 &  769.0 \\
mean  &     35.6 &                       72.7 &     40.0 &                       84.1 &    \textbf{{\color{red}112.2}} &                     \textbf{{\color{red} 294.2}} &    \textbf{172.0} &                      \textbf{467.9} &    \textbf{276.5} &                      \textbf{748.6} &              248.9 &                   663.4 &  499.3 \\
std   &     14.1 &                       42.6 &     16.7 &                       52.8 &     66.8 &                      188.4 &     91.5 &                      251.3 &    112.9 &                      286.9 &               158.0 &                   417.5 &  124.1 \\
\bottomrule
\end{tabular}

\label{selForceWave2Contr2M-20Tab}
%\caption{a caption}
\end{table}


\end{frame}

%%%%%%%%%%%%%%%%%%%%%%%%%%%%%%%%%%%%%%%%%%%%%%%%%%%%%%%%%
\begin{frame}{Fragile persons and future considerations}

\begin{itemize}

\item A quite impressive analysis in Nature, 16 FEBRUARY 2021, 
The coronavirus is here to stay---here's what that means

\medskip

\url{https://www.nature.com/articles/d41586-021-00396-2}

\medskip

\emph{A Nature survey shows many scientists expect the virus that causes COVID-19 to become endemic, but it could pose less danger over time}.

\medskip

\item
If Nature is right, a possible long-term strategy is to stop all fragile people for a given period when $R_t$ starts increasing (fragile workers in sick leave, if unable to work remotely.

\medskip

With social benefits, e.g., schooling, and economic benefits, if activities do not stop

\medskip

The same strategy would also have been surprisingly efficient now, for the second wave.

\medskip

Besides, mainly for the fragile people, adopt prevention and vaccinations.
\end{itemize}

\end{frame}



%5%%%%%%%%%%%%%%%%%%%%%%%%%%%%%%%%%%%%%%%%%%%%%%%%%%%%%%%%
\begin{frame}{Sec. w., new infect. from outs., stop fragile people. 60  days from Oct. 5\footnote{Schools are always working 100\% in this case.}}

%selectResults10kStableSeedsCPoints_basicControlB_schoolOpenSeptNoFragOCT05-60dControlChangingWorldNewStart_plusHMlog.ipynb
% using 10kCtrl1NStartNoFragOCT05-60d.csv from SIsaR_0.9.5.4.1trials10kCtrl1NStartNoFragOCT05-60d.nlogo

\textbf{1407} {\tiny epidemics stable in Summer 2020 out of 10,000, rule: at Jun~1,~20 select if sym. (10, 70], actual v. 33.3 \& at Sep~20,~20 select if sym. (20, 90], actual value 37.5;} \textbf{886} {\tiny at Dec~15,~20, rule: sym.+asym.>Sep~20,~20, actual value: 200.0.}

\begin{figure}[H]
\center
\includegraphics[scale=0.17]{10kForceWave2NoFragOct5for60d.png}
\caption{First wave with non-ph, cont, meas., forcing the sec. w.; \textbf{in sec. w., uniquely stop fragile people, including fragile workers}} 
\label{selForceWave2NoFrag}
\end{figure}

%\vspace{-0.4cm} 

\begin{table}[H]
\center
\tiny
\begin{tabular}{p{0.3cm}p{0.3cm}p{0.3cm}p{0.3cm}p{0.3cm}p{0.3cm}p{0.3cm}p{0.3cm}p{0.3cm}p{0.3cm}p{0.3cm}p{0.3cm}p{0.3cm}p{0.4cm}}
\toprule
(1000) &  Jun~1,~20 & &  Sep~9,~20 & & Dec~15,~20 & & Feb~1,~21 & & May~1,~21 & & Dec~15,~20~~~to~~~end   \\
{} &  sym. &  all &  sympt. &  totalInf. &  sympt. &  totalInf. &  sympt. &  totalInf. &  sympt. &  totalInf. &  sympt. &  totalInf.  & days\\
\midrule
count &   1407.0 &                     1407.0 &   1407.0 &                     1407.0 &    886.0 &                      886.0 &    761.0 &                      761.0 &    637.0 &                      637.0 &              886.0 &                   886.0 &  886.0 \\
mean  &     35.6 &                       72.7 &     40.0 &                       84.1 &    \textbf{{\color{cyan}128.1}} &                      \textbf{{\color{cyan}326.3}} &    \textbf{211.0} &                      \textbf{555.1} &    \textbf{323.3} &                      \textbf{862.1} &               301.1 &                   792.3 &  515.5 \\
std   &     14.1 &                       42.6 &     16.7 &                       52.8 &     89.6 &                      234.2 &    118.1 &                      306.7 &    126.4 &                      315.9 &               170.7 &                   450.2 &  116.9 \\
\bottomrule
\end{tabular}

\label{selForceWave2NoFragTab}
%\caption{a caption}
\end{table}


\end{frame}




%%%%%%%%%%%%%%%%%%%%%%%%%%%%%%%%%%%%%%%%%%%%%%%%%%%%%%%%%
\begin{frame}{To recap}

\begin{table}[H]
\center
\footnotesize
\begin{tabular}{p{1.8cm}p{0.5cm}p{0.5cm}p{0.5cm}p{0.5cm}p{0.5cm}p{0.5cm}}
\toprule
Scenarios     &    &  Dec~15,~20 &               & Dec~15,~20~~~to~~~end  &  &\\
                     &    & sympt.           &  totalInf. &  sympt. &  totalInf.  & days   \\                               

\midrule
no \\
containments in   & count & 140.0 &                      140.0 &                140.0 &                   140.0 &  140.0 \\
spontaneous         &  mean  &  \textbf{248.4} &                      \textbf{648.7} &      701.1 &                  1757.9 &  594.2 \\
second wave         & std  &  167.4 &                      424.3 &               246.4 &                   599.7 &  118.9 \\

\midrule
no \\
containments in   & count &     1044.0 &                     1044.0 &             1044.0 &                  1044.0 & 1044.0 \\
forced         & mean  &       \textbf{180.4} &                      \textbf{462.1} &          726.6 &                  1810.9 &  620.9 \\
second wave         & std   & 134.6 &                      354.6 &           221.9 &                   544.0 &  110.8 \\

\midrule
basic \\
containements in   & count &     874.0 &                      874.0 &          874.0 &                   874.0 &  874.0 \\
forced                    & mean  &             \textbf{130.0} &                      \textbf{340}.6 &                 252.7 &                   666.4 &  494.1 \\
second wave         & std   &   83.9 &                      232.6 &                  156.8 &                   416.4 &  122.7 \\
         
\midrule
-20 days \\ 
containments in   & count &  769.0 &                      769.0 &               769.0 &                   769.0 &  769.0 \\
forced             & mean  &   \textbf{{\color{red}112.2}} &        \textbf{{\color{red} 294.2}} &       248.9 &        663.4 &  499.3 \\
second wave  & std   &   66.8 &                      188.4 &          158.0 &                   417.5 &  124.1 \\

\midrule
frag. p. \& workers \\  
control in              & count &   886.0 &                      886.0 &              886.0 &                   886.0 &  886.0 \\
forced                  & mean  &  \textbf{{\color{cyan}128.1}} &         \textbf{{\color{cyan}326.3}} &          301.1 &        792.3 &  515.5 \\
second wave       & std   &  89.6 &        234.2 &   170.7 &          450.2 &  116.9 \\


\bottomrule
\end{tabular}
\caption{Report of the key results, with count, mean, and std}
\label{keyResultsT}
\end{table}



\end{frame}

%%%%%%%%%%%%%%%%%%%%%%%%%%%%%%%%%%%%%%%%%%%%%%%%%%%%%%%%%
\section{Planning a vaccination campaign}

%%%%%%%%%%%%%%%%%%%%%%%%%%%%%%%%%%%%%%%%%%%%%%%%%%%%%%%%%
\begin{frame}{Planning a vaccination campaign}

\begin{itemize}
\item Exploring vaccination sequences, using \emph{genetic algorithms}. A detailed note, frequently updated, is at \url{https://terna.to.it/simul/GAresultPresentation.pdf}.

\bigskip

\item We compare the effect of choosing vaccination quotas via GA with two predetermined strategies.

\bigskip

\item Key dates: 
\begin{itemize}
\item in the internal calendar of the model, day 373 is Feb. \nth{12}, 2021, which is effectively the starting point of the vaccinations in the region; 

\item the day of the effectiveness of the initial vaccinations, 40 days later, is day 413 (Mar. \nth{22}, 2021).
\end{itemize}

\end{itemize}

\end{frame}

%%%%%%%%%%%%%%%%%%%%%%%%%%%%%%%%%%%%%%%%%%%%%%%%%%%%%%%%%
\begin{frame}{Vaccination groups}

We take into consideration seven groups in order of decreasing fragility but also considering the exposure to contagion:

\begin{enumerate}
\item [\emph{g1}]
	extra fragile people with three components;
	\begin{itemize}
		\item due to intrinsic characteristics: people in nursing homes;
		\item due to risk exposure:
		\begin{itemize}
			\item nursing homes operators;
			\item healthcare operators;
 		\end{itemize} 
 	\end{itemize}  
\item [\emph{g2}]
	teachers;
\item [\emph{g3}]
	workers with medical fragility;
\item [\emph{g4}]
	regular workers;
\item [\emph{g5}]
	fragile people without special characteristics;
\item [\emph{g6}]
	regular people, not young, not worker, and not teacher;
\item [\emph{g7}]
	young people excluding special activity cases (a limited number in \emph{g1}).
\end{enumerate}

\end{frame}

%%%%%%%%%%%%%%%%%%%%%%%%%%%%%%%%%%%%%%%%%%%%%%%%%%%%%%%%%
\begin{frame}{Vaccination quotas, \emph{plain} strategy}

Considering the \emph{plain} option adopted in Table \ref{quotaTable1} and remembering that the time-sequence in daily actions is the winner, we will primarily vaccinate the left column groups to move gradually to other columns: (\emph{g1}) extra fragile people, (\emph{g2}) teachers, (\emph{g3}) fragile workers, (\emph{g4}) regular workers, (\emph{g5}) fragile people, (\emph{g6}) regular people, (\emph{g7}) young people.

\begin{table}[H]
\centering
\begin{tabular}{ccccccccc}
\toprule
From day & \begin{tabular}[c]{@{}c@{}}Q. of \\ vaccines \\ (000)\end{tabular} & \emph{g1} & \emph{g2} & \emph{g3} & \emph{g4} & \emph{g5} & \emph{g6} & \emph{g7} \\
\midrule
373      & 5           & 0.1        & 0.1        & 0.1        & 0.1        & 0.1        & 0.1        & 0.1        \\
433      & 10         & 0.1        & 0.1        & 0.1        & 0.1        & 0.1        & 0.1        & 0.1        \\
493      & 10         & 0.1        & 0.1        & 0.1        & 0.1        & 0.1        & 0.1        & 0.1        \\
553      & 10         & 0.1        & 0.1        & 0.1        & 0.1        & 0.1        & 0.1        & 0.1        \\
613      & 20         & 0.1        & 0.1        & 0.1        & 0.1        & 0.1        & 0.1        & 0.1        \\
738      & end     \\ 
\bottomrule  
\end{tabular}
\caption{From the day of the first column, considering the quantity of the second column (000), the vaccination of each group follows the quota of the related columns}
\label{quotaTable1}
\end{table}

\begin{table}[H]
\centering
\begin{small} % or footnotesize, scriptsize, tiny, etc.
\begin{tabular}{lccccccc}
\toprule
(000) & \emph{g1} & \emph{g2} & \emph{g3} & \emph{g4} & \emph{g5} & \emph{g6} & \emph{g7} \\
\midrule
Susc. at t = 0                   & 133 & 84 & 240 & 1560 & 1179 & 254 & 900 \\
\bottomrule  
\end{tabular}
\end{small}
\caption{Susceptible persons at the beginning of the simulation}
\label{susceptible}
\end{table}


\end{frame}

%%%%%%%%%%%%%%%%%%%%%%%%%%%%%%%%%%%%%%%%%%%%%%%%%%%%%%%%%
\begin{frame}{Vaccination quotas, \emph{wise} strategy}

Considering the \emph{wise} option adopted in Table \ref{quotaTable2} and remembering that the time-sequence in daily actions is the winner, we will primarily vaccinate the left column groups to move gradually to other columns, but postponing group \emph{g4} (regular workers), \emph{g6} (regular people), and \emph{g7} (young people).

\begin{table}[H]
\centering
\begin{tabular}{ccccccccc}
\toprule
From day & \begin{tabular}[c]{@{}c@{}}Q. of \\ vaccines \\ (000)\end{tabular} & \emph{g1} & \emph{g2} & \emph{g3} & \emph{g4} & \emph{g5} & \emph{g6} & \emph{g7} \\
\midrule
373      & 5           & 0.1        & 0.1        & 0.1        & 0.0        & 0.1        & 0.0        & 0.0        \\
433      & 10         & 0.1        & 0.1        & 0.1        & 0.0        & 0.1        & 0.0        & 0.0        \\
493      & 10         & 0.1        & 0.1        & 0.1        & 0.1        & 0.1        & 0.1        & 0.1        \\
553      & 10         & 0.1        & 0.1        & 0.1        & 0.1        & 0.1        & 0.1        & 0.1        \\
613      & 20         & 0.1        & 0.1        & 0.1        & 0.1        & 0.1        & 0.1        & 0.1        \\
738      & end     \\ 
\bottomrule  
\end{tabular}
\caption{From the day of the first column, considering the quantity of the second column (000), the vaccination of each group follows the quota of the related columns}
\label{quotaTable2}
\end{table}

\begin{table}[H]
\centering
\begin{small} % or footnotesize, scriptsize, tiny, etc.
\begin{tabular}{lccccccc}
\toprule
(000) & \emph{g1} & \emph{g2} & \emph{g3} & \emph{g4} & \emph{g5} & \emph{g6} & \emph{g7} \\
\midrule
Susc. at t = 0                   & 133 & 84 & 240 & 1560 & 1179 & 254 & 900 \\
\bottomrule  
\end{tabular}
\end{small}
\caption{Susceptible persons at the beginning of the simulation}
\label{susceptible2}
\end{table}


\end{frame}

%%%%%%%%%%%%%%%%%%%%%%%%%%%%%%%%%%%%%%%%%%%%%%%%%%%%%%%%%
\begin{frame}{Two cases}

\begin{figure}[H]
\center
\includegraphics[scale=0.14]{CaseForGA_I_base.png}~~~\includegraphics[scale=0.14]{CaseForGA_II_base.png}
\caption{Case I~~~~~~~~~~~~~~~~~~~~~~~~~~~~~~~~~~~~~~~~~~~~~~~~~~~~~~~~~~~~~~~~~~~~~~~~~~~~~~~~~~~~~~~~~~~~~~~Case II\\Crucial dates: blue line for the starting point of the vaccination campaign and red line for the start of the effectiveness of the initial vaccinations}
\label{twoCases}
\end{figure}

\end{frame}

%%%%%%%%%%%%%%%%%%%%%%%%%%%%%%%%%%%%%%%%%%%%%%%%%%%%%%%%%
\begin{frame}{GA quotas, an example in case I, with vaccinated people still spreading the infection}

\medskip

\begin{table}[H]
\centering
\begin{small} % or footnotesize, scriptsize, tiny, etc.
\begin{tabular}{lccccccc}
\toprule
& \emph{g1} & \emph{g2} & \emph{g3} & \emph{g4} & \emph{g5} & \emph{g6} & \emph{g7} \\
\midrule
Susc. at t = 0                   & 133 & 84 & 240 & 1560 & 1179 & 254 & 900 \\
Susc. when\\vacc. starts & 124 & 81 & 162 & 1234 & 1032 & 245 & 891 \\
\bottomrule  
\end{tabular}
\end{small}
\caption{Case I: susceptible persons at the beginning of the simulation and when the vaccination campaign starts,  day 373, Feb. \nth{12}, 2021}
\label{susceptibleCaseForGA_I}
\end{table}

Groups: (\emph{g1}) extra fragile people, (\emph{g2}) teachers, (\emph{g3}) fragile workers, (\emph{g4}) regular workers, (\emph{g5}) fragile people, (\emph{g6}) regular people, (\emph{g7}) young people.

\medskip

\begin{table}[H]
\centering
\begin{small} % or footnotesize, scriptsize, tiny, etc.
\begin{tabular}{cclllllll}
\toprule
From day & \begin{tabular}[c]{@{}c@{}}Q. of \\ vaccines \\ (000)\end{tabular} & \emph{g1} & \emph{g2} & \emph{g3} & \emph{g4} & \emph{g5} & \emph{g6} & \emph{g7} \\
\midrule
373 & 5 & 0.01 & 0 & 0 & 0.79 & 0.18 & 0.38 & 0.19 \\
433 & 10 & 0.94 & 0.06 & 0.32 & 0.54 & 0.19 & 0.83 & 0.5 \\
493 & 10 & 0.97 & 0.97 & 0.74 & 0.79 & 0.2 & 0.14 & 0.52 \\
553 & 10 & 0.98 & 0.83 & 0.02 & 0.39 & 0.99 & 0.04 & 0.48 \\
613 & 20 & 0.52 & 0.01 & 0.83 & 0.6 & 1 & 0.27 & 0.9 \\
738      & end   \\ 
\bottomrule  
\end{tabular}
\end{small}
\caption{GA best strategy in case I, with \emph{vaccinated people still spreading the infection}: from the day of the first column, considering the quantity of the second column, the vaccination of each group follows the quota of the related columns}
\label{quotaTableCaseForGA_I_1>413bestGA}
\end{table}


\end{frame}

%%%%%%%%%%%%%%%%%%%%%%%%%%%%%%%%%%%%%%%%%%%%%%%%%%%%%%%%%
\begin{frame}{Synopsys}

Three hypotheses about contagion transmission from vaccinate people if infected: 100\%, never, 50\%.

\medskip

\begin{table}[H]
\centering
\begin{scriptsize} % or footnotesize, scriptsize, tiny, etc.
\begin{tabular}{llllllllllll}
\toprule
Case    & At   & Final      & Final        & Final       & Final      & Final       & Final    & Final        & Final    & Final        & Final     \\
             & day & no         & plain       & wise        & GA         & plain       & wise        & GA         & plain       & wise        & GA    \\
             & 413 & vaccin. & vaccin.    & vaccin.  & vaccin.  & vaccin.  & vaccin.  & vaccin.  & vaccin.  & vaccin.  & vaccin.  \\
 (1000) &        &              &  infect.  &  infect. &  infect. &  infect. &  infect. &  infect. &  infect. &  infect. &  infect.\\
             &       &              &  100\%   &  100\% &  100\% &  0\% &  0\% &  0\% & 50\% &  50\% &  50\% \\
\midrule
I            & 197 & 325 & 236 & 263 & \textbf{200} & 203 & 211   & 199 & 204 & 229 & 203 \\
             & -      & 128 & 39    & 66  & \textbf{3}    &  6     & 14  & 2     & 7     & 32 & 6 \\
\\
II           & 233 & 375 & 355 &  344 & \textbf{305}  & 340 & 334 & \textbf{297}  & 356 & 344 &  \textbf{288} \\
             & -      &  142 & 122 & 111 & \textbf{72}    & 107 & 101 & \textbf{64}   & 123   & 111 &  \textbf{55} \\
\\
\bottomrule  
\end{tabular}
\end{scriptsize}
\caption{Results of the campaigns in the two cases, only symptomatic people (second row in each case: minus day 413)}
\label{caseSynopsys}
\end{table}

\end{frame}
%%%%%%%%%%%%%%%%%%%%%%%%%%%%%%%%%%%%%%%%%%%%%%%%%%%%%%%%%
\section{A new model}

%%%%%%%%%%%%%%%%%%%%%%%%%%%%%%%%%%%%%%%%%%%%%%%%%%%%%%%%%
\begin{frame}{A new model: the map}

\begin{figure}[H]
\center
\includegraphics[scale=0.25]{Piem1.png}~~~~~~~~~\includegraphics[scale=0.25]{Piem2.png} 

\caption{3D Piedmont} 
\label{Piem}
\end{figure}

\end{frame}

%%%%%%%%%%%%%%%%%%%%%%%%%%%%%%%%%%%%%%%%%%%%%%%%%%%%%%%%%
\begin{frame}{A new model: the scale and the items}

\begin{itemize}

\item $1:100$.

\item \textit{Infection engine}, \url{https://terna.to.it/simul/InfectionEngine.pdf}.

\item Houses.
\item Schools.
\item Hospitals.
\item Nursing homes,
\item Factories.
\color{red}
\item Transportations.
\item Aggregation places: happy hours, night life, sport stadiums, discotheques, \ldots
\color{blue}
\item Networks (family networks, professional networks, high-contact individuals,\footnote{G. Manzo and A. van de Rijt. Halting sars-cov-2 by targeting high-contact individuals. Journal of Artificial Societies and Social Simulation, 23(4):10, 2020. ISSN 1460-7425. doi: 10.18564/jasss.4435. URL http://jasss.soc.surrey.ac.uk/23/4/10.html.} \ldots) 

\end{itemize}

\end{frame}

%%%%%%%%%%%%%%%%%%%%%%%%%%%%%%%%%%%%%%%%%%%%%%%%%%%%%%%%%
\begin{frame}{The tool: S.L.A.P.P.}

Scientific advertising: \url{https://terna.github.io/SLAPP/}

\begin{figure}[H]
\center
\includegraphics[scale=0.26]{SLAPP.png}

\caption{Swarm-Like Agent Protocol in Python} 
\label{SLAPP}
\end{figure}

\end{frame}
%%%%%%%%%%%%%%%%%%%%%%%%%%%%%%%%%%%%%%%%%%%%%%%%%%%%%%%%%
\section{Final remarks}

%%%%%%%%%%%%%%%%%%%%%%%%%%%%%%%%%%%%%%%%%%%%%%%%%%%%%%%%%
\begin{frame}{Some final considerations}

\begin{itemize}

\item
The importance of High Performance Computing.

\bigskip

\item The S.I.s.a.R. model is a tool for comparative analyses, not for forecasting (the enormous standard deviation values are intrinsic to the problem).


\item The model is highly parametric and more it will be.

\item New crisis calling for immediate simulation could take a substantial advantage from the parametric structure of the model.

\end{itemize}

 \bigskip
 The slides are at \url{https://terna.to.it/simul/TernaCCA20210329.pdf}.
 
 My homepage \url{https://terna.to.it} and my mail address \url{pietro.terna@unito.it}.
 
\end{frame}


\end{document}